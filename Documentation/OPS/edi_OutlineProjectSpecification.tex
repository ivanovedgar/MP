\documentclass[11pt,fleqn,twoside]{article}
\usepackage{makeidx}
\makeindex
\usepackage{palatino} %or {times} etc
\usepackage{plain} %bibliography style 
\usepackage{amsmath} %math fonts - just in case
\usepackage{amsfonts} %math fonts
\usepackage{amssymb} %math fonts
\usepackage{lastpage} %for footer page numbers
\usepackage{fancyhdr} %header and footer package
\usepackage{mmpv2} 
\usepackage{url}

% the following packages are used for citations - You only need to include one. 
%
% Use the cite package if you are using the numeric style (e.g. IEEEannot). 
% Use the natbib package if you are using the author-date style (e.g. authordate2annot). 
% Only use one of these and comment out the other one. 
\usepackage{cite}
%\usepackage{natbib}

\begin{document}


\name{Edgar Ivanov}
\userid{edi}
\projecttitle{Stabilisation of a Pan-and-Tilt Unit holding a camera}
\projecttitlememoir{Stabilisation of a Pan-and-Tilt Unit holding a camera} %same as the project title or abridged version for page header
\reporttitle{Outline Project Specification}
\version{0.1}
\docstatus{Draft}
\modulecode{CS39440}
\degreeschemecode{G421}
\degreeschemename{Ubiquitos Computing}
\supervisor{Frederic Labrosse} % e.g. Neil Taylor
\supervisorid{FFL}
\wordcount{}

%optional - comment out next line to use current date for the document
%\documentdate{10th February 2014} 
\mmp

\setcounter{tocdepth}{3} %set required number of level in table of contents


%==============================================================================
\section{Project description}
%==============================================================================
My final year project will contain both hardware and software parts. The aim of the project is to build an additional unit which will ensure that camera mounted on the rover is always in a stable position. Currently there is a Pan-and-Tilt Unit (PTU) which controls the camera. It uses gyroscopes to acquire the current position in the space and then adjusts the position of the camera. However gyroscopes are not perfect and tend to drift over the time, as well as providing wrong data at the different temperatures. 

I will be building an additional unit which will acquire current position using the more accurate inclinometer and then will provide the PTU with the calibration data. Essential my task is to write a program which will interact with the PTU and inclinometer. The program fill fetch data from both devices and compare it. Based on the difference between PTU and inclinometer information it will calculate by how much degrees PTU should be adjusted and then will send this calibration data to the PTU. At the end there should be a simple interface that allows to perform cali bration by just sending one command from the unit that controls the rover.   

%==============================================================================
\section{Proposed tasks}
%==============================================================================
Tasks for this project may be divided in to two parts: the first part is about the hardware and the second is about the software. 

\subsection{Hardware tasks}
Essentially there will be three pieces of hardware used in this project: the Raspberry Pi (RPi) computer, PTU and inclinometer, there also will be some relays to turn on and off PTU as well as GPIO14 chip to convert data from the analogue to digital format. RPi will be used as a platform for the software to run on and it will be connected to the PTU and inclinometer to control them. Pan-and-Tilt Unit will be communicating with RPi using serial connection and inclinometer will be connected (throughout GPIO14 chip) using \begin{math}I^2 C\end{math} bus. For this part of the project I will need to do some research about the RPi, PTU and inclinometer to understand how they should be properly wired to make them work. In particular it means reading data sheets for these devices, examining connection diagrams, understanding commands.

\subsection{Software tasks}
I was given the C++ code which has implemented functionality to control the PTU, so my first task will be to make sure that this code can be compiled and run on the RPi computer. However this implementation is missing one, essential for this project, piece of code which calibrates the PTU. My second task will be to find what functions are to stabilize with drift compensation and implement them in the code. Then I can treat this code as a complete module and make calls to it when I need. The third task is to write code that takes readings from the inclinometer. This data then will be used to calibrate PTU. Following tasks will be to write server code which connects everything together and respond to the calibration requests from the main computer.

%==============================================================================
\section{Project deliverables}
%==============================================================================
During the course of this project the will by multiple deliverables, the most obvious are: outline project specification, mid-project demonstration, final report and final demonstration.

% Start to comment out / remove the following lines. They are only provided for instruction for this example template.  You don't need the following section title, because it will be added as part of the bibliography section. 
%
%==============================================================================
\section*{Your Bibliography - REMOVE this title and text for final version}
%==============================================================================
%
You need to include an annotated bibliography. This should list all relevant web pages, books, journals etc. that you have consulted in researching your project. Each reference should include an annotation. 

The purpose of the section is to understand what sources you are looking at.  A correctly formatted list of items and annotations is sufficient. You might go further and make use of bibliographic tools, e.g. BibTeX in a LaTeX document, could be used to provide citations, for example \cite{NumericalRecipes} \cite{MarksPaper} \cite[99-101]{FailBlog} \cite{kittenpic_ref}.  The bibliographic tools are not a requirement, but you are welcome to use them.   

You can remove the above {\em Your Bibliography} section heading because it will be added in by the renewcommand which is part of the bibliography. The correct annotated bibliography information is provided below. 
%
% End of comment out / remove the lines. They are only provided for instruction for this example template. 
%


\nocite{*} % include everything from the bibliography, irrespective of whether it has been referenced.

% the following line is included so that the bibliography is also shown in the table of contents. There is the possibility that this is added to the previous page for the bibliography. To address this, a newline is added so that it appears on the first page for the bibliography. 
\newpage
\addcontentsline{toc}{section}{Initial Annotated Bibliography} 

%
% example of including an annotated bibliography. The current style is an author date one. If you want to change, comment out the line and uncomment the subsequent line. You should also modify the packages included at the top (see the notes earlier in the file) and then trash your aux files and re-run. 
%\bibliographystyle{authordate2annot}
\bibliographystyle{IEEEannot}
\renewcommand{\refname}{Annotated Bibliography}  % if you put text into the final {} on this line, you will get an extra title, e.g. References. This isn't necessary for the outline project specification. 
\bibliography{mmp} % References file

\end{document}
