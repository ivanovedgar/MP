\chapter{Introduction}
This section should discuss your preparation for the project, including background reading, your analysis of the problem and the process or method you have followed to help structure your work.  It is likely that you will reuse part of your outline project specification, but at this point in the project you should have more to talk about. 

I was interested in getting experience with assembling and programming real world sensors and mini computers where low level programming would be involved. This project attracted my attention because is was solving real world problem, having a real customer and contained both hardware and software parts.

\section{Problem Definition}
Pan-and-Tilt Unit (PTU), which is mounted on the Idris electric vehicle, is used to hold a panoramic camera. This vehicle is used for the research and mainly drives over the curved surfaces. PTU that holds a camera has a built in functionality to stabilize at the chosen position, so that if rover drives up the hill camera stays vertical. Theoretically to achieve camera verticality rover could be standing on the flat surface when stabilization command is issued, or its position could be adjusted manually if it is on a curved surface and then stabilization command issued. To ensure stability of the camera PTU uses gyroscopes, which in a perfect world would be enough. Unfortunately the well known problem of the gyroscopes is that their readings are affected by the different air temperature, magnetic effects, friction etc.\cite{JacobFraden2010}. In reality when PTU is issued with the stabilization command it slowly drifts on both axes and in 1-2 minutes goes to the position which is 20 degrees different from where is should be. 

One of the possible solutions could be the use of the additional PTU functionality which allows to cancel drift by specifying drift rate in radians per second calculated in advance. The problem with this solution is that drift rate can change dramatically if surround environment changes (sun goes behind the clouds and air temperature drops affecting gyroscope readings) and manually calculating new drift rate every 10 minutes becomes infeasible and tedious. 

Proposed solution is to automate drift rate calculation so that it can be adjusted while driving. However this approach requires additional information about the current PTU orientation in the space (its inclination angles with respect to gravity). For this purpose can be used electronic inclinometer. It is a special type of the accelerometer, inclinometer response is electric signal representative of an angle between the internal axis and the gravity vector \cite{JacobFraden2010}. Accurate reading with such device can be obtained only if it is in a stable position and does not accelerate, otherwise it will provide erroneous data representing sum of two vectors – earth gravity and acceleration. This imposes certain limitations as to of when PTU drift rate calibration can be preformed, meaning that the vehicle will have to stop every time to perform recalibration.  

The aim of the project is to build an additional unit. Essential my task is to write a program which will interact with the PTU and inclinometer. The program will fetch data from both devices and compare it. Based on the difference between PTU and inclinometer information it will calculate by how much degrees PTU should be adjusted and then will send this calibration data to the PTU. At the end of the project there should be a simple interface that allows to perform calibration by just sending one command from the unit that controls the rover.

\section{Background}
What was your background preparation for the project? What similar systems did you assess? What was your motivation and interest in this project? 

   

\section{Analysis}
Taking into account the problem and what you learned from the background work, what was your analysis of the problem? How did your analysis help to decompose the problem into the main tasks that you would undertake? Were there alternative approaches? Why did you choose one approach compared to the alternatives? 

There should be a clear statement of the objectives of the work, which you will evaluate at the end of the work. 

In most cases, the agreed objectives or requirements will be the result of a compromise between what would ideally have been produced and what was felt to be possible in the time available. A discussion of the process of arriving at the final list is usually appropriate.

\section{Process}
You need to describe briefly the life cycle model or research method that you used. You do not need to write about all of the different process models that you are aware of. Focus on the process model that you have used. It is possible that you needed to adapt an existing process model to suit your project; clearly identify what you used and how you adapted it for your needs.

