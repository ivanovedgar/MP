\chapter{Introduction}
This section should discuss your preparation for the project, including background reading, your analysis of the problem and the process or method you have followed to help structure your work.  It is likely that you will reuse part of your outline project specification, but at this point in the project you should have more to talk about. 

I was interested in getting experience of assembling and programming real world sensors and mini computers where low level programming would be involved. This project attracted my attention because is was solving real world problem having real customer and contained both hardware and software parts.

\section{Problem Definition}
Pan-and-Tilt Unit, which is mounted on the Idris electric vehicle, is used to hold a panoramic camera. The vehicle is used for the research and mainly drives over the curved surface.  PTU that holds a camera has a built in functionality to stabilize on the given position. This feature together with the inclinometer providing data about the current chassis position can be used to ensure camera verticality when rover goes up or down the hill. 


The aim of the project is to build an additional unit which will ensure that camera mounted on the rover is always in a stable position. Currently there is a Pan-and-Tilt Unit (PTU) which holds the camera. It uses gyroscopes to acquire the current position in the space and adjust its position. However gyroscopes are not perfect and tend to drift over the time, as well as providing wrong data at different temperatures.

I will be building an additional unit which will acquire current position using the more accurate inclinometer and then will provide the PTU with the calibration data. Essential my task is to write a program which will interact with the PTU and inclinometer. The program will fetch data from both devices and compare it. Based on the difference between PTU and inclinometer information it will calculate by how much degrees PTU should be adjusted and then will send this calibration data to the PTU. At the end of the project there should be a simple interface that allows to perform calibration by just sending one command from the unit that controls the rover.

\section{Background}
What was your background preparation for the project? What similar systems did you assess? What was your motivation and interest in this project? 

   

\section{Analysis}
Taking into account the problem and what you learned from the background work, what was your analysis of the problem? How did your analysis help to decompose the problem into the main tasks that you would undertake? Were there alternative approaches? Why did you choose one approach compared to the alternatives? 

There should be a clear statement of the objectives of the work, which you will evaluate at the end of the work. 

In most cases, the agreed objectives or requirements will be the result of a compromise between what would ideally have been produced and what was felt to be possible in the time available. A discussion of the process of arriving at the final list is usually appropriate.

\section{Process}
You need to describe briefly the life cycle model or research method that you used. You do not need to write about all of the different process models that you are aware of. Focus on the process model that you have used. It is possible that you needed to adapt an existing process model to suit your project; clearly identify what you used and how you adapted it for your needs.

